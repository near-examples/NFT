\documentclass[11pt,titlepage]{article}
\usepackage{TCSToolkit}

\newcommand{\dofloor}[1]{\lfloor#1\rfloor}
\newcommand{\doceil}[1]{\lceil#1\rceil}


%%%%% Stuff you can change %%%%%%%%%%%%%%%%%%%%%%%%%%%%%%%%%%
\newcommand{\myname}{Lev Stambler}
%


%
% Final tip: you can reference HW5 in your TeX using \Cref{hw:5}; 
% and, you can reference HW5.Problem3 in your TeX using \Cref{prob:5.3}
%


%%%%% Section-renaming code by egreg
\makeatletter
% we use \prefix@<level> only if it is defined
\renewcommand{\@seccntformat}[1]{%
  \ifcsname prefix@#1\endcsname
    \csname prefix@#1\endcsname
  \else
    \csname the#1\endcsname\quad
  \fi}
% Now we define our homework section prefixes
% \newcommand\prefix@section{Homework \thesection: }
% \newcommand{\prefix@subsection}{Problem \thesubsection: }
% \newcommand{\prefix@subsubsection}{Section \thesubsubsection: }
\makeatother
%%%%%




\begin{document}

\title{Drawing Math}

\author{\myname}

\date{\today}

\maketitle

\pagebreak
\section{Introduction}
- Numberphile video
- I was bored and this looked fun
- I wasn't paying attention in one of my lectures...
- This paper is purely for funs... for now

\section{Background}
- Digital math
- Euler's formula
- I payed attention in some of my lectures

\section{Definitions and questions}
\subsection{Definitions}
Say you (yes you!) had a turtle living in $D$ dimensional Euclidean
space and in discrete time. At time step $i$, where $i \in \Z$ and $i > 0$,
the turtle has position $p_i \in \R^D$.
Then, lets define $\Delta p_{i+1} = p_{i+1} - p_i$; in other words, $\Delta p_{i+1}$ is the change in position from time $i$ to $i + 1$.

Now say that the turtle's movement is determined by $k$ seed parameters drawn from
the same set. Then, for some state space $\mathcal{S}$,
define $s_i^j \in \mathcal{S}$ to be some arbitrary
state associated with timestamp $i$ for the $j$th seed parameter where $j \in [k]$.
Next we will define a set of functions $SU_j: \mathcal{S} \rightarrow \mathcal{S}$
(for $S$tate $U$pdater) such that $s_{i + 1}^j = SU_j(s_i^j, i)$. Note that for
$j, a \in [m]$ where $j \neq a$, $s_{i+1}^j$ is determined solely by $s_i^j$ and $i$ and not
$s_i^a$.

Now that we have our machinery built up, lets define $f^k: \mathcal{S} \rightarrow \R$
and $Comb: \R^k \rightarrow \R^d$ such that
$$
  \Delta p_{i + 1} = Comb\left(f^1(s^1_{i + 1}), f^2(s^2_{i + 1}), ..., f^k(s^k_{i + 1})\right).
$$
In other words, $Comb$ takes in a real number determined by the state of each seed and
returns an update to the position of the turtle.

\subsection{Our case}
We only consider the case where
$$
  Comb(x_1, x_2, ..., x_k) = \left(\prod_{i=1}^k x_j ^ {\mathrm{incl}_1^j}, ..., \prod_{i=1}^k x_j ^ {\mathrm{incl}_D^j}\right)
$$
where $incl_d^j \in \set{0, 1}$ for $d \in [D]$ indicates whether to include a given
$x \in R$ determined by seed $j$ for position update in the $d$th dimension.



\section{Results}
\cite{watrous2018theory}

\section{Conclusion}

\section{Open Questions}


\section*{Acknowledgments}


\bibliographystyle{alpha}
\bibliography{bib/ref}


\end{document}
